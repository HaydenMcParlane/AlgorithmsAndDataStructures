\documentclass{article}
\usepackage{algpseudocode}
\usepackage{algorithm}
\usepackage{algorithmicx}
\usepackage{mathtools}
\usepackage[T1]{fontenc}
\usepackage{amsmath, amsfonts}
\usepackage{enumitem}
\usepackage{blkarray}
\usepackage{tabu}
\usepackage{tikz}

\DeclarePairedDelimiter\ceil{\lceil}{\rceil}
\DeclarePairedDelimiter\floor{\lfloor}{\rfloor}

\newcommand\encircle[1]{
    \tikz[baseline=(X.base)]
        \node (X) [draw, shape=circle, inner sep=0]{\strut #1};
}

\begin{document}

% \begin{titlepage}

%     \begin{center}
%         \large\textbf{CS 5592: Design and Analysis of Algorithms} \\
%         \large\textbf{Mid-term Make Up} \\        
%         \large\textit{Author: Hayden McParlane}
%     \end{center}
% \end{titlepage}


% \encircle{F.3} 
% Given a sequence of n (n > 2) real numbers in an array, A[1], A[2],
% A[3], ..., A[n], we wish to find two numbers A[i] and A[j], where i < j, such that
% A[i] \(\leq\) A[j] and their distance is the largest. That is, 
% \begin{flalign*}
%     j - i &= \max\{ v - u \mid 1 \leq u < v \leq n \text{ and } A[u] \leq A[v]\}
% \end{flalign*}
% If no such pair, report -\(\infty\). Please design an \(\mathcal{O}(n \cdot lg n)\) algorithm
% to solve this problem. You can use any method , including divide and conquer, greedy
% or dynamic programming.
% \begin{flalign*}
%     &\\
% \end{flalign*}
% \textbf{The following should be noted about the algorithm below (in addition to
% the statements in the problem):}
% \begin{enumerate}
%         \item Function \Call{SortAndTrackIndices}{A, I} returns array A sorted in
%         non-decreasing order \((a_1 \leq a_2 \leq a_3 \leq ... \leq a_n)\) as well
%         as an array, \(I\), which maps each \(a_i\) to its original position, \(I_i\).        
% \end{enumerate}

% \begin{algorithm}[H]
% \caption{
%     Greedy solution to the problem above. Make sure to take note of the assumptions
%     above.        
% }
% \label{alg:algorithm-label}
% \begin{algorithmic}[1]
% \Function{MaxDist}{A, n}        % low to high approach
%     \State $ A, I \gets \Call{SortAndTrackIndices}{A} $    
%     \State $ i_{max} \gets j_{max} \gets I[n] $
%     \State $ i \gets j \gets I[n] $    
%     \\
%     \For{\(k\) from \(n\) - 1 to 1}
%         \If{\(j < I[k]\)}
%             \State $ i \gets I[k] $     
%             \State $ j \gets I[k] $                   
%         \Else
%             \If{\(i > I[k]\)}                
%                 \State $ i \gets I[k] $ 
%             \EndIf            
%         \EndIf
%         \If{\(j - i > j_{max} - i_{max}\)}
%             \State $ i_{max} \gets i $
%             \State $ j_{max} \gets j $            
%         \EndIf
%     \EndFor
%     \\
%     \If{\(i_{max}\) equals \(j_{max}\)}
%         \State $ \Return \text{ -}\infty $
%     \Else
%         \State $ \Return \textit{ i} \text{ and } \textit{j} $
%     \EndIf
%     \\    
% \EndFunction

% \end{algorithmic}
% \end{algorithm}

\pagebreak
\encircle{F.4} Consider the following activity selection problem. Suppose we have a lecture hall
which is available from time t = 0. There are n activities, \(a_1, a_2, a_3, ..., a_n\), that apply
for using the lecture hall. If the application of \(a_i (1 \leq i \leq n) \) is accepted, then it
must start before or at \(s_i (\geq 0)\) and it must continuously use the hall for \(t_i\) time.
Moreover, at any time, at most one activity is allowed to use the lecture hall. We wish to select 
a set R of activities such that they can be arranged to use the lecture hall with no time conflict
and the total time the lecture hall is used by these activities is maximized,
\[
    \sum_{a_i \in R}{t_i} \text{ is maximized.}
\]

\begin{enumerate}[label=\Alph*]
    \item Re-define the problem as a decision problem.
    \item Polynomially reduce the set partition problem to the decision problem
    in the first part.
\end{enumerate}

\textbf{F.4.A} Because this optimization problem requests that we maximize the total time
the lecture hall is used our restatement of the problem will focus on time usage of the
hall. Therefore, the problem can be \textbf{restated as a decision problem in the following manner}:
\\
\\
Suppose we have a lecture hall
which is available from time t = 0. There are n activities, \(a_1, a_2, a_3, ..., a_n\), that apply
for using the lecture hall. If the application of \(a_i (1 \leq i \leq n) \) is accepted, then it
must start before or at \(s_i (\geq 0)\) and it must continuously use the hall for \(t_i\) time.
Moreover, at any time, at most one activity is allowed to use the lecture hall. 
\textbf{Can we select a set of compatible activities such that they occupy the
 lecture hall for a duration of k units of time?}
\\

\textbf{F.4.B} Transformation of the set partition problem into an instance of this problem
can be done using the following strategy.\\ \\
The set partition problem is defined as follows: Given a set, \(S = \{v_1, v_2, v_3, ..., v_n\}\),
can we partition that set into two subsets such that
\begin{flalign*}
    \sum_{v_i \in A}{v_i} &= \sum_{v_i \in S - A}{v_i}
\end{flalign*}
Following this transformation algorithm we can accomplish that task:
\begin{enumerate}
    \item Compute \(M = \sum_{v_i \in S}{v_i}\).
    \item We can assume any number \(v_i\) in S is less than or equal to M/2. Otherwise,
    no partition is possible and the transformation is trivial.
    \item For each \(v_i \in A\), construct an activity such that:
    \begin{flalign*}
        t_i &= v_i &\\
        s_i &= (M/2) - v_i
    \end{flalign*}
    \item For each \(v_i \in S - A\), construct an activity such that:
    \begin{flalign*}
        t_i &= v_i &\\
        s_i &= M - v_i
    \end{flalign*}
    \item Set \(k = M\).
\end{enumerate}

\textbf{An explanation of logic follows. Each numer in the list below corresponds to the same numbered
step above.}
\begin{enumerate}
    \item We compute the total sum of all values in S because that value is needed in steps that follow.
    \item We assume all numbers are less than or equaled to M/2 because if that's not the case a valid
    partition cannot be done satisfying the requirements of the set partition problem. Such sets simply
    result in the answer No, and are therefore trivial.
    \item For all \(v_i \in A\), we transform \(v_i\) directly into \(t_i\) because it is a sensical
    approach to this transformation. The more interesting aspect of this step is transformation of
    \(v_i\) into \(s_i\) by subtracting \(v_i\) from \((M/2)\). This is necessary because we can view
    this subset of S as occurring in the first half of the time during which we have scheduled the 
    lecture hall. All activities that result from this transformation will occupy the lecture hall at
    some instance between time \(t_0\) and time \((M/2)\).
    \item This step is necessarily separated from the previous step because we can view all of the
    activities that take place in this subset as occuring after all of those that came from the 
    subset A. All activities in the subset A occurred between \(t_0\) and \((M/2)\), whereas all
    activities that result from this step will occur between time \((M/2)\) and \(M\). Therefore,
    activities that are part of this second set, \(S - A\), have \(s_i\) calculated by subtracting
    \(v_i\) from \(M\).
    \item Finally, we need to set our k value. In our new problem, k is to represent the total
    duration during which the lecture hall is occupied. Because \(M = \sum_{v_i \in S}{v_i}\) and
    \(v_i\) was transformed to \(t_i\) directly and without operation (i.e, they are one-to-one),
    \(M = \sum_{v_i \in S}{v_i} = \sum_{t_i \in A}{t_i} + \sum_{t_i \in S-A}{t_i} = \) the total
    duration during which the lecture hall will be occupied. This value is, therefore, our desired
    \(k\), which we defined as being the duration during which the lecture hall will be occupied.
\end{enumerate}

\end{document}