\documentclass{article}
\usepackage{algpseudocode}
\usepackage{algorithm}
\usepackage{algorithmicx}
\usepackage{mathtools}
\usepackage[T1]{fontenc}
\usepackage{amsmath, amsfonts}
\usepackage{enumitem}
\usepackage{blkarray}
\usepackage{tabu}
% \usepackage{tabularx}
\usepackage{tikz}

\newcommand\encircle[1]{
    \tikz[baseline=(X.base)]
        \node (X) [draw, shape=circle, inner sep=0]{\strut #1};
}

\begin{document}

\begin{titlepage}

    \begin{center}
        \large\textbf{CS 5592: Design and Analysis of Algorithms} \\
        \large\textbf{Homework 5} \\
        \large\textit{Author: Hayden McParlane}
    \end{center}

\end{titlepage}

% \encircle{1} <problem statement>

\encircle{1.a} Prove that this decision problem is in the NP-class. \\ \\
Proof that a decision problem is in the NP-class simply requires writing a
verification algorithm that can verify a given certificate in polynomial
time.

\begin{algorithm}[H]
    \caption{
        A verification algorithm for the problem above. This algorithm assumes
        that the projects in need of completion, P, are known to keep its format
        consistent with typical NP-class verification algorithms.
    }
    \label{alg:algorithm-label}
    \begin{algorithmic}[1]
        \Function{Verify}{C, k}
            \For{project in P}
                \For{company in k}
                    \State $ \text{1. Retrieve projects from L completable by the current company.} $
                    \State $ \text{2. Remove the project from the project list.} $
                \EndFor
            \EndFor
            \If{project list is empty}
                \State $ \Return \text{ Yes} $
            \Else
                \State $ \Return \text{ No} $
            \EndIf
        \EndFunction
    \end{algorithmic}
\end{algorithm}

This algorithm proves that the problem is in the NP-class because it executes the
verification in polynomial time. For each project chosen, all of the companies
are checked to determine which projects that company can complete. If a company is
found that can complete every project, the algorithm returns yes. Otherwise, no is
returned. This is polynomially related to the number of projects in P because the
problem states that all of the projects can be completed but that none of the
companies can complete all of the projects alone. This means that the upper bound
on the list of projects associated with a given company is \(|P| = m\). So,
even if each company could complete all of the projects the verification have the upper
bound \(|P| \cdot (!|P|)^k \leq |P| \cdot |P|^k = |P|^{k+1} = m^{k+1} = m^{constant}\).
Therefore, this problem is no harder than the NP-class and belongs to the NP-class.\\

% \encircle{1.b} Prove that this decision problem is NP-hard. (Hint: reduce the vertex 
% cover problem to this problem.) \\

% TODO: \\

\encircle{1.c} Suppose you have an algorithm \Call{A}{P, m, C, n, k} that can
determine, for any given integer k, whether k companies are enough to handle all 
projects. (It gives a "yes" or "no" answer.). Please show how we can use the
algorithm \Call{A}{P, m, C, n, k} to select the smallest number k of companies
to finish all m projects so we can maximally save our budget. You need to output
the companies that have been selected by your method (algorithm). \\

\begin{algorithm}[H]
    \caption{
        An algorithm that uses the given function to find
        the minimum cost. This algorithm assumes that the cost of each
        company is the same.
    }
    \label{alg:algorithm-label}
    \begin{algorithmic}[1]
        \Function{FindMinCost}{P, m, C, n}                        
            \State $ k \gets n $            
            \Repeat
                \State $ k \gets k - 1 $                
            \Until{\Call{A}{P, m, C, n, k} equals NO}            
            \State $ k \gets k + 1 $            
            
            \While{ \Call{sizeOf}{C} > 0}
                \State $ current \gets \Call{pop}{C} $
                \If{ \Call{A}{P, n, C, n, k} equals NO}
                    \State $ C \gets current \bigcup{} C $
                    \State $ \text{break from while} $                                                            
                \EndIf
            \EndWhile 
            \State $ \Return \text{ C} $
        \EndFunction
    \end{algorithmic}
\end{algorithm}

This algorithm starts off by giving k a value of n. The problem statement
states that n companies will be able to complete all of the projects so we
know the result of a call to A will be YES giving us the ability to skip
that call. Each iteration of the repeat decrements k until the call to A
returns No at which point the loop breaks. When this occurs, the current
value of k received a No so we can assume that the smallest k would be the
previous k value; k + 1. Therefore, k is assigned k + 1. \\

The while-loop following that assignment ensures that companies are still
present in C. If there are companies present, a next company is popped off
of C at which point another call to A is done to determine if removal of
that company from C resulted in No. During the while portion, k is kept
constant while the elements of C vary. When the call to A is done and 
results in no in the while-loop, we can assume that the previously eliminated
company was vital and needs to be in the solution.

\pagebreak
\encircle{2} Consider the following activity selection problem. Suppose we 
have two lecture halls which are available from time t = 0. There are n activities, 
\(a_1, a_2, a_3, ..., a_n\), that apply for using either one of these two
lecture halls. Each activity \(a_i\), \((1 \leq i \leq n)\) has flexible
starting time but must start before or at \(s_i\) \((\geq 0)\) and requests to
use the hall for \(t_i\) hours. We wish to select the largest set of scheduled
activities such that their requests can be satisfied with no time conflict.
That is, they can be scheduled into one of these two lecture halls such that, 
at any time, at most one activity takes place in each hall.
\begin{enumerate}[label=\Alph*]
    \item Re-define this problem as a decision problem.
    \item Prove that this decision problem is NP-hard.
\end{enumerate}

\textbf{2.A} Re-defining this problem as a decision problem can be done as
follows.

% Consider the following activity selection problem. Suppose we 
% have two lecture halls which are available from time t = 0. There are n activities, 
% \(a_1, a_2, a_3, ..., a_n\), that apply for using either one of these two
% lecture halls. Each activity \(a_i\), \((1 \leq i \leq n)\) has flexible
% starting time but must start before or at \(s_i\) \((\geq 0)\) and requests to
% use the hall for \(t_i\) hours. \textbf{We wish to select the largest set of scheduled
% activities such that their requests can be satisfied with no time conflict.}
% That is, they can be scheduled into one of these two lecture halls such that, 
% at any time, at most one activity takes place in each hall.

% Here, the sentence "We wish to select the largest set of scheduled activities
% such that their requests can be satisfied with no time confict" is key in 
% the restatement of this problem. Because determination of the maximum can be
% done by testing increasing the size of a theoretical activity set gradually
% and checking if that set is doable, we can take the optimization statement
% above and re-write it as "Can we find a set of k activities such that all k
% activities can be scheduled in the two lecture halls without time conflict?".

% \textbf{Therefore, the total problem statement becomes: }

Consider the following activity selection problem. Suppose we 
have two lecture halls which are available from time t = 0. There are n activities, 
\(a_1, a_2, a_3, ..., a_n\), that apply for using either one of these two
lecture halls. Each activity \(a_i\), \((1 \leq i \leq n)\) has flexible
starting time but must start before or at \(s_i\) \((\geq 0)\) and requests to
use the hall for \(t_i\) hours. \textbf{Can we find a set of k activities such that all k
activities can be scheduled in the two lecture halls without time conflict?}

\text{}\\
\textbf{2.B} A fairly intuitive problem to map to our new problem above in this
proof is the set partion problem.

The set partition problem is defined as follows. Given a set of numbers,
\(S = \{a_1, a_2, a_3, ..., a_n\}\), can we partition S into two sets, A and B,
such that the following is true:
\[
    \sum_{i \in A}{v_i} = \sum_{i \in B}{v_i}
\]
It's important to realize that in order for the above to be true, the total sum
of every element in the set \(S\) must be \(M\) meaning that each partition must
sum to \(M/2\). That is,
\begin{flalign*}
    \sum_{i \in S}{v_i} &= M &\\
\end{flalign*}

We further assume that all elements of the set S have a value less than M/2.
If this weren't the case partitioning the set wouldn't be possible which is
an extraneous case not in need of consideration here. With this in
mind, we can map the set partition problem to that above in the following manner:
\begin{enumerate}
    \item Take each set value, \(v_i\), and map it to a corresponding duration
    \(t_i\).
    \item Take each set value, \(v_i\), and map it to a latest starting time
    \(s_i\) using the following equation:
    \begin{flalign*}
        s_i &= (M/2) - t_i &\\
    \end{flalign*}
    \item Add two new values, \(a_{n+1}\) and \(a_{n+2}\) such that 
    \begin{flalign*}
        a_{n+1} &= 
    \end{flalign*}
\end{enumerate}
\end{document}