\documentclass{article}
\usepackage{algpseudocode}
\usepackage{algorithm}
\usepackage{algorithmicx}
\usepackage{mathtools}
\usepackage[T1]{fontenc}
\usepackage{amsmath, amsfonts}
\usepackage{tikz}

\newcommand\encircle[1]{
    \tikz[baseline=(X.base)]
        \node (X) [draw, shape=circle, inner sep=0]{\strut #1};
}

\begin{document}

\begin{titlepage}

    \begin{center}
        \large\textbf{CS 5592: Design and Analysis of Algorithms} \\
        \large\textbf{Homework 5} \\
        \large\textit{Author: Hayden McParlane}
    \end{center}

\end{titlepage}

\encircle{1} <problem statement>

\encircle{1.a} Prove that this decision problem is in the NP-class. \\ \\
Proof that a decision problem is in the NP-class simply requires writing a
verification algorithm that can verify a given certificate in polynomial
time.

\begin{algorithm}[H]
    \caption{
        A verification algorithm for the problem above. This algorithm assumes
        that the projects in need of completion, P, are known to keep its format
        consistent with typical NP-class verification algorithms.
    }
    \label{alg:algorithm-label}
    \begin{algorithmic}[1]
        \Function{Verify}{C, k}
            \For{project in P}
                \For{company in k}
                    \State $ \text{1. Retrieve projects from L completable by the current company.} $
                    \State $ \text{2. Remove the project from the project list.} $
                \EndFor
            \EndFor
            \If{project list is empty}
                \State $ \Return \text{ Yes} $
            \Else
                \State $ \Return \text{ No} $
            \EndIf
        \EndFunction
    \end{algorithmic}
\end{algorithm}

This algorithm proves that the problem is in the NP-class because it executes the
verification in polynomial time. For each project chosen, all of the companies
are checked to determine which projects that company can complete. If a company is
found that can complete every project, the algorithm returns yes. Otherwise, no is
returned. This is polynomially related to the number of projects in P because the
problem states that all of the projects can be completed but that none of the
companies can complete all of the projects alone. This means that the upper bound
on the list of projects associated with a given company is \(|P| = m\). So,
even if each company could complete all of the projects the verification would be
bound by \(|P| \cdot |P|^k = |P|^{k+1} = m^{k+1} = m^{constant}\). TODO: Verify proof!!!

\encircle{1.b} Prove that this decision problem is NP-hard. (Hint: reduce the vertex 
cover problem to this problem.) \\


\encircle{1.c}

\end{document}