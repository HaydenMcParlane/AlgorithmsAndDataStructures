\documentclass{article}
\usepackage{algpseudocode}
\usepackage{algorithm}
\usepackage{algorithmicx}
\usepackage{mathtools}
\usepackage[T1]{fontenc}
\usepackage{amsmath, amsfonts}

\DeclarePairedDelimiter\ceil{\lceil}{\rceil}
\DeclarePairedDelimiter\floor{\lfloor}{\rfloor}
\begin{document} 

    \section{Pseudocode Algorithm}

    \begin{algorithm}[H]             
        \caption{Find max two such that: \[i < j\] and \[A[i] \leq A[j]\]}
        \label{alg:algorithm-label}
        \begin{algorithmic}[1]
            \Function{FindMax}{A, p, r}
                \If{ p equals r}
                    \Return $ (-\infty, -\infty) $ 
                \Else
                    \State $ mid \gets \floor{(p + r)/2} $
                    \State $ leftSolution \gets \Call{FindMax}{A, p, mid} $
                    \State $ rightSolution \gets \Call{FindMax}{A, mid + 1, r} $
                    \State $ crossingSolution \gets \Call{FindCrossingMax}{A, p, mid, r} $
                    \State $ \Return $ \Call{Max}{leftSolution, crossingSolution, rightSolution} $ $
                \EndIf
            \EndFunction
        \end{algorithmic}        
    \end{algorithm}    

    \begin{algorithm}[H]               
        \caption{Perform work to actually find the max two for a given p and r.}
        \label{alg:algorithm-label}
        \begin{algorithmic}[1]
            \Function{FindCrossingMax}{A, p, mid, r}                
                \State $ i \gets \text{mid} $
                \State $ j \gets \text{mid} + 1 $            
                \State $ max \gets  A[i] + A[j]$    
                \For{k from mid down to p}
                    \If{ max < A[k] +  A[j]}
                        \State $ i \gets k $
                        \State $ max \gets A[k] + A[j] $
                    \EndIf
                \EndFor

                \For{k from mid + 1 up to r}
                    \If{ max < A[k] + A[i] }
                        \State $ j \gets k $
                        \State $ max \gets A[k] + A[i] $
                    \EndIf
                \EndFor

                \State $ \Return $ (\textit{i, j})
            \EndFunction
        \end{algorithmic}        
    \end{algorithm}    
        
    \section{Complexity Analysis ** REVISIT **}

    This analysis starts off by realizing that this algorithm is recursive.
    We already know that being that it is implemented following the 
    divide-and-conquer paradigm. However, it is worth proper definition.
    The recursive nature of this approach can be broken down as follows:

    \[ \Call{T}{n} = \Call{T}{\floor{n/2}} + \Call{T}{\ceil{n/2}} + \Call{T}{n} + \mathcal{O}(1) \]

    Because of the monotonically increasing nature of the complexity function (i.e, 
    values are only being added and not subtracted) this can be simplified for easy
    comprehension. We can assume
    \[ n = 2^k \]
    as well as eliminating the ceiling and floor function allowing for the following 
    substitution:
    \[ 
        \Call{T}{2^k} = \Call{T}{2^k/2} + \Call{T}{2^k/2} + \Call{T}{n} + \mathcal{O}(1) 
                      = \Call{T}{2^{k-1}} + \Call{T}{2^{k-1}} + \Call{T}{n} + \mathcal{O}(1)
                      = 2 \cdot \Call{T}{2^{k-1}} + \Call{T}{n} + \mathcal{O}(1)
    \]
    Additionally, if we substitute
    \[ \Call{W}{k} = \Call{T}{2^k} \]
    then further simplification can come about, as shown below.
    \[ 
        \Call{W}{k} = \Call{W}{k - 1} + \Call{W}{k - 1} + \mathcal{O}(1)
                    = 2 \cdot \Call{W}{k - 1} + \mathcal{O}(1)
    \]
    Using the summation method of recurrence resolution we can solve the 
    recurrence above.
    \begin{flalign*}
        \Call{W}{k} &= 2 \cdot \Call{W}{k - 1} + \mathcal{O}(1) &\\
        \Call{W}{k-1} &= 2 \cdot \Call{W}{k - 2} + \mathcal{O}(1) &\\
        \Call{W}{k-2} &= 2 \cdot \Call{W}{k - 3} + \mathcal{O}(1) &\\
        \Call{W}{k-3} &= 2 \cdot \Call{W}{k - 4} + \mathcal{O}(1)        
    \end{flalign*}

    Therefore,    

    \begin{flalign*}
        \Call{W}{k} &= 2 \cdot \Call{W}{k - 1} + \mathcal{O}(1) &\\
                    &= 2 \cdot ( 2 \cdot \Call{W}{k - 2} + \mathcal{O}(1) ) + \mathcal{O}(1) &\\
                    &= 2^2 \cdot \Call{W}{k - 2} + 2 \cdot \mathcal{O}(1) + \mathcal{O}(1)   &\\
                    &= 2^2 \cdot (2 \cdot \Call{W}{k - 3} + \mathcal{O}(1)) + 2 \cdot \mathcal{O}(1) 
                    + \mathcal{O}(1) &\\
                    &= 2^3 \cdot \Call{W}{k - 3} + 2^2 \cdot \mathcal{O}(1) + 2 \cdot \mathcal{O}(1)
                    + \mathcal{O}(1) &\\
                    &= 2^3 \cdot ( 2 \cdot \Call{W}{k - 4} + \mathcal{O}(1) ) + 2^2 \cdot \mathcal{O}(1) + 2 \cdot \mathcal{O}(1)
                    + \mathcal{O}(1) &\\
                    &= 2^4 \cdot \Call{W}{k - 4} + 2^3 \cdot \mathcal{O}(1) + 2^2 \cdot \mathcal{O}(1) + 2 \cdot \mathcal{O}(1)
                    + \mathcal{O}(1)
    \end{flalign*}                

    In general, we can see that i recursive calls contributes
    \[ \Call{W}{i} = 2^i \cdot \Call{W}{k - i} + \sum_{n=0}^{i-1} 2^n \cdot \mathcal{O}(1) \]
    to the overall complexity. Using the base case as \(\Call{W}{1} = 1\) resulting from the
    execution of one comparison operation, we can solve the recurrence.

    \begin{flalign*}
        \Call{W}{k - 1} &= 2^{k - 1} \cdot \Call{W}{k - (k - 1)} + \sum_{n=0}^{k-2} 2^n \cdot \mathcal{O}(1) &\\                    
                        &= 2^{k - 1} \cdot \Call{W}{1} + \sum_{n=0}^{k-2} 2^n &\\
                        &= 2^{k - 1} \cdot (1) + \sum_{n=0}^{k-2} 2^n &\\
                        &= 2^{k - 1} + \sum_{n=0}^{k-2} 2^n &\\
                        &\leq 2^{k - 1} + \sum_{n=0}^{k-1} 2^n
    \end{flalign*}

    The last summation shows us an upper bound including a geometric series
    which further simplifies to

    \begin{flalign*}
        \Call{W}{k-1} &\leq 2^{k - 1} + \sum_{n=0}^{k-1} 2^n &\\
                      &= 2^{k - 1} + \frac{2^{(k-1) + 1} - 1}{2 - 1} &\\
                      &= 2^{k - 1} + 2^k - 1 &\\
                    %   &= \mathcal{O}(2^k)
    \end{flalign*}

    Remembering that we set \(n = 2^k\) and solving for k:
    \begin{flalign*}
        n = 2^k \implies k = \log_2 n
    \end{flalign*}

    Finally, substituting k for n:
    \begin{flalign*}
        \Call{T}{n} &= 2^{k - 1} + 2^k - 1 &\\
                    &= 2^{(\log_2 n) - 1} + 2^{log_2 n} - 1
                    &= 
    \end{flalign*}
    

\end{document}