\documentclass{article}
\usepackage{algpseudocode}
\usepackage{algorithm}
\usepackage{algorithmicx}
\usepackage{mathtools}
\usepackage[T1]{fontenc}
\usepackage{amsmath, amsfonts}
\usepackage{tikz}

\newcommand\encircle[1]{
    \tikz[baseline=(X.base)]
        \node (X) [draw, shape=circle, inner sep=0]{\strut #1};
}

\begin{document}

\begin{titlepage}

    \begin{center}
        \large\textbf{CS 5592: Design and Analysis of Algorithms} \\
        \large\textbf{Homework 2} \\        
        \large\textit{Author: Hayden McParlane}
    \end{center}

\end{titlepage}

\encircle{1} There are n houses located on a west-to-east street. H[i] (meters), \(1 \leq i \leq n\),
is the distance from the west end of the street to the ith house. You may assume that H[1] < H[2] <
H[3] < ... < H[n]. There is no post office on the street. We plan to build several post offices on the
street such that any house can reach a post office within 100 meters. Please design an \(\mathcal{O}(n)\)
algorithm to compute the locations for the post offices, P[j] (meters), \(1 \leq j \leq m\), where P[j]
is the distance from the west end of the street to the jth post office. Make sure that the number of 
post offices, m, is minimized.

\begin{algorithm}[H]
    \caption{
        Algorithm used to place the minimal number of post offices
        such that each house in H will have a post office at most
        100 meters away.
    }
    \label{alg:algorithm-label}
    \begin{algorithmic}[1]
        \Function{PlacePostOffices}{H, n}
            \State $ P \gets \emptyset $
            \\
            \State $ \Call{Append}{P, H[1] + 100} $
            \State $ j \gets 1 $
            \For{i from 2 to n}
                \If{|H[i] - P[j]| > 100}
                    \State $ \Call{Append}{P, H[i] + 100} $
                    \State $ j \gets j + 1 $
                \EndIf
            \EndFor
            \\
            \State $ \Return \text{ P} $
        \EndFunction
    \end{algorithmic}
\end{algorithm}

\pagebreak
\encircle{2} Suppose we drive a pickup truck from city A to city B. Along the highway we will
pass through n apple markets labeled with 1, 2, 3, ..., n, where you can buy or sell apples.
City A and city B also have an apple market each. From a customer point of view, the buying
price B[i] and selling price S[i] (dollar per pound) at market i are known. Now, we will stop
at one of the stationsto buy apples and then stop and another station to sell apples. Please
design an \(\mathcal{O}(n)\) greedy algorithm to find market i to buy apples and market j\(\geq\)
i to sell apples such that the profit will be maximized. We assume that it would be too costly
and forbidden to drive backward. You need to do exactly one trade even if the profit is negative.

\begin{algorithm}[H]
    \caption{
        Greedy algorithm used maximize profit for the apple buy/sell
        problem above. Note that in certain instances this algorithm
        may be suboptimal, a quality common to greedy algorithms.
    }
    \label{alg:algorithm-label}
    \begin{algorithmic}[1]
        \Function{MaxProfit}{B, S, n}
            \State $ buy \gets 0 $
            \State $ sell \gets 0 $            
            \For{i from 1 to n + 1}
                \If{B[buy] > B[i]}
                    \State $ buy \gets i $                    
                    \State $ sell \gets i $
                \Else
                    \If{S[sell] < S[i]}
                        \State $ sell \gets i $
                    \EndIf                     
                \EndIf                                
            \EndFor            
            \\
            \State $ \Return \text{ (buy, sell)} $

        \EndFunction
    \end{algorithmic}
\end{algorithm}

\pagebreak
\encircle{3} Re-consider the following problem in quiz 1: Given a sequence of n real numbers
stored in an array, A[1], A[2], A[3], ..., A[n], we wish to find two numbers A[i] and A[j],
where i < j, such that A[i] \(\leq\) A[j] and their sum is the largest. If no such two numbers
exist, report -\(\infty\). This time, please design an \(\mathcal{O}(n)\) greedy algorithm
to solve this problem.

\begin{algorithm}
    \caption{
        Greedy algorithm solving the problem above. It's important to note 
        that a quality of greedy algorithms is possible suboptimal performance.
        This solution may perform suboptimally whereas the divide-and-conquer 
        algorithm presented previously would not. However, this algorithm performs
        better. It is a trade-off. 
    }
    \label{alg:algorithm-label}
    \begin{algorithmic}
        \Function{LargestSum}{A, n}
            \State $ max \gets -\infty $ 
            \State $ submax \gets -\infty $ 
            \\            
            \For{k from 1 to n}
                \If{max \(\leq\) A[k]}
                    \State $ subMax \gets max $
                    \State $ max \gets A[k] $                                    
                \EndIf
            \EndFor            
            \\
            \If{max equals -\(\infty\) or submax equals -\(\infty\)}
                \State $ \Return \text{ -}\infty $
            \Else
                \State $ \Return \text{ \textit{max} and \textit{submax}} $
            \EndIf                            
        \EndFunction
    \end{algorithmic}
\end{algorithm}

\end{document}