\documentclass{article}
\usepackage{algpseudocode}
\usepackage{algorithm}
\usepackage{algorithmicx}
\usepackage{mathtools}
\usepackage[T1]{fontenc}
\usepackage{amsmath, amsfonts}
\usepackage{enumitem}
\usepackage{blkarray}
\usepackage{tabu}
% \usepackage{tabularx}
\usepackage{tikz}

\newcommand\encircle[1]{
    \tikz[baseline=(X.base)]
        \node (X) [draw, shape=circle, inner sep=0]{\strut #1};
}

\begin{document}

% \begin{titlepage}

%     \begin{center}
%         \large\textbf{CS 5592: Design and Analysis of Algorithms} \\        
%         \large\textbf{Homework 3} \\        
%         \large\textit{Author: Hayden McParlane}
%     \end{center}

% \end{titlepage}

% \encircle{1} Find an optimal parenthesization of a matrix chain multiplication whose
% sequence of dimensions is (7, 10, 9, 5, 12, 6). \\

% We begin by naming the matrices which concern us. We will assume

% \begin{flalign*}
%     \textbf{A} \text{ is 7 x 10 matrix} \\    
%     \textbf{B} \text{ is 10 x 9 matrix} \\
%     \textbf{C} \text{ is 9 x 5 matrix} \\
%     \textbf{D} \text{ is 5 x 12 matrix} \\
%     \textbf{E} \text{ is 12 x 6 matrix} \\
% \end{flalign*}

% As is customary in dynamic programming, we will use a results
% matrix to hold solutions that are found during the process.
% We will also create a second matrix that will store the position
% of the parenthesis in order to retrace our results and return
% the optimal solution to the problem itself. We will call this
% matrix \textbf{S}.

% Initialization of our initial matrix will be done and will zero
% all of the 

% \[
% \begin{blockarray}{cccccc}
% \textbf{A} & \textbf{B} & \textbf{C} & \textbf{D} & \textbf{E} \\
% \begin{block}{(ccccc)c}
%   0 & 0 & 0 & 0 & 0 & \textbf{E} \\
%   0 & 0 & 0 & 0 & 0 & \textbf{D} \\
%   0 & 0 & 0 & 0 & 0 & \textbf{C} \\
%   0 & 0 & 0 & 0 & 0 & \textbf{B} \\
%   0 & 0 & 0 & 0 & 0 & \textbf{A} \\
% \end{block}
% \end{blockarray}
%  \]

\encircle{2} Suppose n activities apply for using a common resource. 
Activity \(a_i\) \((1 \leq i \leq n)\) has a starting time \(S[i]\) 
and a finish time \(F[i]\) such that \(0 < S[i] < F[i]\). Two activities
\(a_i\) and \(a_j\) \((1 \leq i, j \leq n)\) are compatible if intervals
[\(S[i], F[i]\)) and [\(S[j], F[j]\)) do not overlap. We assume the 
activities have been sorted such that \(S[1] \leq S[2] \leq ... \leq S[n]\).
\begin{enumerate}[label=\Alph*]
    \item Design an \(\mathcal{O}(n^2)\) dynamic programming algorithm
    to find a set of compatible activities such that the total amount
    of time the resource is used by these compatible activities is
    maximized. You need to define the sub-problems, establish the
    inductive formula and show the initial conditions. Pseudocode is
    not required.
    \item Apply your algorithm to the following set of activities:
    \begin{center}
        \begin{tabular}{ | l | c | c | c | c | c | c | c | c | c | c | r |}
            \hline
            i & 1 & 2 & 3 & 4 & 5 & 6 & 7 & 8 & 9 & 10 & 11 \\ \hline
            S[i] & 2 & 3 & 5 & 6 & 7 & 9 & 10 & 12 & 13 & 14 & 16 \\ \hline
            F[i] & 6 & 5 & 7 & 10 & 8 & 13 & 16 & 14 & 14 & 18 & 20 \\ 
            \hline
        \end{tabular}
    \end{center}    
\end{enumerate}

\pagebreak
\textbf{2.A}

\begin{algorithm}[H]
    \caption{
        A dynamic programming algorithm usable to solve the activity problem above
        in \(\mathcal{O}(n^2)\) time. In this algorithm...
    }
    \label{alg:algorithm-label}
    \begin{algorithmic}[1]
        \Function{MaxActivities}{S[1..n], F[1..n]}
            \State $ M \gets P \gets \emptyset $        
            \State $ \Call{Initialize}{M, P} $            
            \\
            \State $ M[1] \gets 1 $
            \State $ P[1] \gets 0 $
            \For{i from 2 to n}   \Comment{Locate max for \(m_i\)}                
                \State $ max \gets 0 $
                \State $ maxIdx \gets 0 $                
                \For{j from 1 to i - 1}
                    \If{max < M[j] and F[j] \(\leq\) S[i]}
                        \State $ max \gets M[j] $
                        \State $ maxIdx \gets j $
                    \EndIf
                \EndFor               
                \State $ M[i] \gets max + 1$ 
                \State $ P[i] \gets maxIdx $
            \EndFor
            \\
            \State $ max \gets 1 $ \Comment{Find global maximum}                        
            \For{i from 2 to n}      
                \If{M[max] < M[i]}
                    \State $ max \gets i $                    
                \EndIf
            \EndFor
            \\
            \State $ \Return \text{ (max, P)}$
            
        \EndFunction
    \end{algorithmic}
\end{algorithm}

With this algorithm in mind, the answer is as follows:
\begin{flalign*}
    \textbf{ Inductive Formula: } \{ m(i) &= \max_{1 \leq j \leq i}(m_j) + 1 \mid F[j] < S[i] \} &\\
    \textbf{Initial Conditions: } m(1) &= 1 &\\    
\end{flalign*}

The \textbf{subproblem} can be thought of as follows. Each activity, \(i\),
will be added to a chain of activities resulting in the maximum set that
includes activity \(i\). Activity \(i\) is always included in its maximum.
Its maximum, \(m_i\), must also be based off of previous activities which
are compatible.
\\

\pagebreak

\textbf{2.B}\\

\textit{After Initialization of M and P.}
\begin{center}
    \begin{tabular}{ | l | c | c | c | c | c | c | c | c | c | c | r |}
        \hline
            i & 1 & 2 & 3 & 4 & 5 & 6 & 7 & 8 & 9 & 10 & 11 \\ \hline
            M[i] & 1 & 0 & 0 & 0 & 0 & 0 & 0 & 0 & 0 & 0 & 0 \\ \hline
            P[i] & 0 & 0 & 0 & 0 & 0 & 0 & 0 & 0 & 0 & 0 & 0 \\
        \hline
    \end{tabular}
\end{center}
\text{}\\

\textit{Upon Algorithm Completion.}
\begin{center}
    \begin{tabular}{ | l | c | c | c | c | c | c | c | c | c | c | r |}
        \hline
            i & 1 & 2 & 3 & 4 & 5 & 6 & 7 & 8 & 9 & 10 & 11 \\ \hline
            M[i] & 1 & 1 & 2 & 2 & 3 & 4 & 4 & 4 & 5 & 6 & 6 \\ \hline
            P[i] & 0 & 0 & 2 & 1 & 3 & 5 & 5 & 5 & 6 & 9 & 9 \\
        \hline
    \end{tabular}
\end{center}    

\pagebreak

\encircle{3}

\begin{algorithm}[H]
    \caption{
        A dynamic programming algorithm usable to solve the chess board
        problem above.
    }
    \label{alg:algorithm-label}
    \begin{algorithmic}[1]
        \Function{MaxCoinValue}{A, V, n, m}
            \State $ P \gets \emptyset $            
            \State $ \Call{Initialize}{P} $ \\            
            \For{i from 1 to n}
                \For{j from 1 to m}
                    \If{ \Call{TopValue}{A, i, j} > \Call{LeftValue}{A, i, j} }
                        \State $ A[i, j] \gets V[i, j] + \Call{TopValue}{A, i, j} $                
                        \State $ P[i, j] \gets \Call{Top}{i, j} $
                    \Else
                        \State $ A[i, j] \gets V[i, j] + \Call{LeftValue}{A, i, j} $
                        \State $ P[i, j] \gets \Call{Left}{i, j} $
                    \EndIf
                \EndFor                
            \EndFor            
            \\            
            \State $ i \gets n $
            \State $ j \gets m $
            \State $ path \gets \emptyset $
            \State $ \Call{Push}{path, (i, j)} $
            \Repeat
                \State $ previous \gets P[i, j] $
                \State $ \Call{Push}{path, previous} $
                \State $ i \gets \Call{IValue}{previous} $
                \State $ j \gets \Call{JValue}{previous} $
            \Until{\Call{IsNill}{P, i, j}}
            \\
            \State $ \Return \text{ path} $
        \EndFunction
    \end{algorithmic}
\end{algorithm}

This algorithm utilizes the following helpers:

\begin{algorithm}[H]
    \caption{
        Returns the value of the square above inputs i and j.
    }
    \label{alg:algorithm-label}
    \begin{algorithmic}[1]
        \Function{TopValue}{A, i, j}
            \If{\Call{IsNotNil}{A[i - 1, j]}}
                \State $ \Return \text{ A[i - 1, j]} $
            \Else
                \State $ \Return \text{ -\infty} $
            \EndIf
        \EndFunction
    \end{algorithmic}
\end{algorithm}

\begin{algorithm}[H]
    \caption{
        Returns the value of the square to the left of inputs 
        i and j.
    }
    \label{alg:algorithm-label}
    \begin{algorithmic}[1]
        \Function{LeftValue}{A, i, j}
            \If{\Call{IsNotNil}{A[i, j - 1]}}
                \State $ \Return \text{ A[i, j - 1]} $
            \Else
                \State $ \Return \text{ -\infty} $
            \EndIf
        \EndFunction
    \end{algorithmic}
\end{algorithm}

and similarly

\begin{algorithm}[H]
    \caption{
        Returns the indices that point to the square atop
        of square (i, j).
    }
    \label{alg:algorithm-label}
    \begin{algorithmic}[1]
        \Function{Top}{i, j}
            \State $ \Return \text{ (i - 1, j)} $
        \EndFunction
    \end{algorithmic}
\end{algorithm}

\begin{algorithm}[H]
    \caption{
        Returns the indices that point to the square sitting left
        of square (i, j).
    }
    \label{alg:algorithm-label}
    \begin{algorithmic}[1]
        \Function{LeftValue}{A, i, j}
            \State $ \Return \text{ (i, j - 1)} $
        \EndFunction
    \end{algorithmic}
\end{algorithm}

\end{document}
