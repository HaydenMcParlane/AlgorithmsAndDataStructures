\documentclass{article}
\usepackage{algpseudocode}
\usepackage{algorithm}
\usepackage{algorithmicx}
\usepackage{mathtools}
\usepackage[T1]{fontenc}
\usepackage{amsmath, amsfonts}
\usepackage{tikz}

\newcommand\encircle[1]{
    \tikz[baseline=(X.base)]
        \node (X) [draw, shape=circle, inner sep=0]{\strut #1};
}

\DeclarePairedDelimiter\ceil{\lceil}{\rceil}
\DeclarePairedDelimiter\floor{\lfloor}{\rfloor}
\begin{document} 

% \begin{titlepage}

%     \begin{center}
%         \large\textbf{CS 5592: Design and Analysis of Algorithms} \\
%         \large\textbf{Homework <num-here>} \\        
%         \large\textit{Author: Hayden McParlane}
%     \end{center}

% \end{titlepage}

\encircle{3} TODO: Problem Statement

    \section{Pseudocode Algorithm}

    \begin{algorithm}[H]             
        \caption{Find max two such that: \[i < j\] and \[A[i] \leq A[j]\]}
        \label{alg:algorithm-label}
        \begin{algorithmic}[1]
            \Function{FindMax}{A, p, r}
                \If{ p equals r}
                    \Return $ (-\infty, -\infty) $ 
                \Else
                    \State $ mid \gets \floor{(p + r)/2} $
                    \State $ leftSolution \gets \Call{FindMax}{A, p, mid} $
                    \State $ rightSolution \gets \Call{FindMax}{A, mid + 1, r} $
                    \State $ crossingSolution \gets \Call{FindCrossingMax}{A, p, mid, r} $
                    \State $ \Return $ \Call{Max}{leftSolution, crossingSolution, rightSolution} $ $
                \EndIf
            \EndFunction
        \end{algorithmic}        
    \end{algorithm}    

    \begin{algorithm}[H]               
        \caption{Perform work to actually find the max two for a given p and r.}
        \label{alg:algorithm-label}
        \begin{algorithmic}[1]
            \Function{FindCrossingMax}{A, p, mid, r}                
                \State $ i \gets \text{mid} $
                \State $ j \gets \text{mid} + 1 $            
                \State $ max \gets  A[i] + A[j]$    
                \For{k from mid down to p}
                    \If{ max < A[k] +  A[j]}
                        \State $ i \gets k $
                        \State $ max \gets A[k] + A[j] $
                    \EndIf
                \EndFor

                \For{k from mid + 1 up to r}
                    \If{ max < A[k] + A[i] }
                        \State $ j \gets k $
                        \State $ max \gets A[k] + A[i] $
                    \EndIf
                \EndFor

                \State $ \Return $ (\textit{i, j})
            \EndFunction
        \end{algorithmic}        
    \end{algorithm}    
        
    \section{Complexity Analysis}

    We can think of this algorithm in terms of three parts: left call, right call
    and a function that finds the answer which crosses over the from the left 
    to the right. With that in mind, the time complexity of this algorithm can
    be modeled using the following equation:

    \begin{flalign*}
        \Call{T}{n} &= \Call{T}{\floor{n/2}} + \Call{T}{\ceil{n/2}} + \Call{f}{n} + \mathcal{O}(1)
    \end{flalign*}

    Recognizing that the time complexity is monotonically increasing and
    is bounded by the functions and not the constant \(\mathcal{O}(1)\), we can
    eliminate the floor and ceiling functions and simplify the constant time 
    element as follows:

    \begin{flalign*}
        \Call{T}{n} &= \Call{T}{n/2} + \Call{T}{n/2} + n + 1 &\\
                    &= 2 \cdot \Call{T}{n/2} + n +  1
    \end{flalign*}

    Using the master method it is evident that
    
    \begin{flalign*}
        a &= 2 &\\
        b &= 2 &\\    
        \Call{f}{n} &= n
    \end{flalign*}

    Therefore

    \begin{flalign*}
        \Call{f}{n} &= \theta(n^{log_b a}) &\\
                    &= \theta(n^{log_2 2}) &\\
                    &= \theta(n^1) &\\
                    &= \theta(n)
    \end{flalign*}

    Implying that we should use rule 2 of the master method. Using rule two,
    we determine the following:

    \begin{flalign*}
        \Call{T}{n} &= \theta(n^{log_b a} \cdot log_2 n) &\\
                    &= \theta(n \cdot log_2 n)
    \end{flalign*}    

\pagebreak
\encircle{4} TODO: Problem Statement

\begin{algorithm}[H]
    \caption{
        TODO        
    }
    \label{alg:algorithm-label}
    \begin{algorithmic}[1]
        \Function{FindDominating}{A, p, r}        
        \If{p equals r}
            \State $ count \gets 1 $
            \State $  $
            \State $ \Return \text{ \textit{count} and \textit{}} $
        \Else
            \State $ threshold \gets \ceil{n/2} $
            \State $ mid \gets \floor{(p + r) / 2} $
            \State $ left \gets \Call{FindDominating}{A, p, mid} $
            \State $ right \gets \Call{FindDominating}{A, mid + 1, r} $
            \\

            \\
            \State $ \Return \text{ } $
        \EndIf

        \EndFunction
    \end{algorithmic}
\end{algorithm}

\pagebreak
\encircle{5} TODO: Problem Statement

\begin{algorithm}[H]
    \caption{
        TODO;
    }
    \label{alg:algorithm-label}
    \begin{algorithmic}[1]
        \Function{MaxSubsequence}{A, p, r}            
            \If{p equals r}                
                \State $ \Return \text{ p, r and A[p]} $
            \Else
                \State $ mid \gets \floor{(p + r)/2} $
                \State $ left \gets \Call{MaxSubsequence}{A, p, mid} $
                \State $ right \gets \Call{MaxSubsequence}{A, mid + 1, r} $
                \State $ crossing \gets \Call{MaxCrossingSubsequence}{A, p, mid, r} $
                \State $ \Return \text{ \(\max(left, crossing, right)\)} $
            \EndIf            
        \EndFunction
    \end{algorithmic}
\end{algorithm}

\begin{algorithm}[H]
    \caption{
        TODO;
    }
    \label{alg:algorithm-label}
    \begin{algorithmic}[1]
        \Function{MaxCrossingSubsequence}{A, p, mid, r}
            \State $ i \gets mid $
            \State $ j \gets mid + 1 $
            \State $ max \gets A[i] + A[j] $
            \State $ sequenceSum \gets max $
            \\            
            \For{k from mid - 1 to p}
                \State $ sequenceSum \gets sequenceSum + A[k] $
                \If{sequenceSum > max}
                    \State $ max \gets sequenceSum $
                    \State $ i \gets k $
                \EndIf
            \EndFor
            \\
            \For{k from mid + 2 to r}
                \State $ sequenceSum \gets sequenceSum + A[k] $
                \If{sequenceSum > max}
                    \State $ max \gets sequenceSum $
                    \State $ j \gets k $
                \EndIf
            \EndFor
            \\
            \State $ \Return \text{ i, j and max} $            
        \EndFunction
    \end{algorithmic}
\end{algorithm}



\end{document}