\documentclass{article}
\usepackage{algpseudocode}
\usepackage{algorithm}
\usepackage{algorithmicx}
\usepackage{mathtools}
\usepackage[T1]{fontenc}
\usepackage{amsmath, amsfonts}
\usepackage{tikz}

\newcommand\encircle[1]{
    \tikz[baseline=(X.base)]
        \node (X) [draw, shape=circle, inner sep=0]{\strut #1};
}

\DeclarePairedDelimiter\ceil{\lceil}{\rceil}
\DeclarePairedDelimiter\floor{\lfloor}{\rfloor}
\begin{document} 

\begin{titlepage}

    \begin{center}
        \large\textbf{CS 5592: Design and Analysis of Algorithms} \\
        \large\textbf{Homework 1} \\        
        \large\textit{Author: Hayden McParlane}
    \end{center}

\end{titlepage}

\encircle{3} Given a sequence of n real numbers stored in an array, A[1], A[2], A[3], ..., A[n],
we wish to find two numbers A[i] and A[j], where i < j, such that A[i] \(\leq\) A[j] and their
sum is the largest. Please design a divide-and-conquer algorithm to solve the above problem.
Please analyze the complexity of your algorithm. The complexity of your algorithm must be
\(\mathcal{O}(n \cdot lg n)\) or better.

    \textbf{Pseudocode}

    \begin{algorithm}[H]             
        \caption{Find max two such that: \[i < j\] and \[A[i] \leq A[j]\]}
        \label{alg:algorithm-label}
        \begin{algorithmic}[1]
            \Function{FindMax}{A, p, r}
                \If{ p equals r}
                    \Return $ (-\infty, -\infty) $ 
                \Else
                    \State $ mid \gets \floor{(p + r)/2} $
                    \State $ leftSolution \gets \Call{FindMax}{A, p, mid} $
                    \State $ rightSolution \gets \Call{FindMax}{A, mid + 1, r} $
                    \State $ crossingSolution \gets \Call{FindCrossingMax}{A, p, mid, r} $
                    \State $ \Return $ \Call{Max}{leftSolution, crossingSolution, rightSolution} $ $
                \EndIf
            \EndFunction
        \end{algorithmic}        
    \end{algorithm}    

    \begin{algorithm}[H]               
        \caption{Perform work to actually find the max two for a given p and r.}
        \label{alg:algorithm-label}
        \begin{algorithmic}[1]
            \Function{FindCrossingMax}{A, p, mid, r}                
                \State $ i \gets \text{mid} $
                \State $ j \gets \text{mid} + 1 $            
                \State $ max \gets  A[i] + A[j]$    
                \For{k from mid down to p}
                    \If{ max < A[k] +  A[j]}
                        \State $ i \gets k $
                        \State $ max \gets A[k] + A[j] $
                    \EndIf
                \EndFor

                \For{k from mid + 1 up to r}
                    \If{ max < A[k] + A[i] }
                        \State $ j \gets k $
                        \State $ max \gets A[k] + A[i] $
                    \EndIf
                \EndFor

                \State $ \Return $ (\textit{i, j})
            \EndFunction
        \end{algorithmic}        
    \end{algorithm}    
        
    \textbf{Complexity Analysis}

    We can think of this algorithm in terms of three parts: left call, right call
    and a function that finds the answer which crosses over the from the left 
    to the right. With that in mind, the time complexity of this algorithm can
    be modeled using the following equation:

    \begin{flalign*}
        \Call{T}{n} &= \Call{T}{\floor{n/2}} + \Call{T}{\ceil{n/2}} + \Call{f}{n} + \mathcal{O}(1)
    \end{flalign*}

    Recognizing that the time complexity is monotonically increasing and
    is bounded by the functions and not the constant \(\mathcal{O}(1)\), we can
    eliminate the floor and ceiling functions and simplify the constant time 
    element as follows:

    \begin{flalign*}
        \Call{T}{n} &= \Call{T}{n/2} + \Call{T}{n/2} + n + 1 &\\
                    &= 2 \cdot \Call{T}{n/2} + n +  1
    \end{flalign*}

    Using the master method it is evident that
    
    \begin{flalign*}
        a &= 2 &\\
        b &= 2 &\\    
        \Call{f}{n} &= n
    \end{flalign*}

    Therefore

    \begin{flalign*}
        \Call{f}{n} &= \theta(n^{log_b a}) &\\
                    &= \theta(n^{log_2 2}) &\\
                    &= \theta(n^1) &\\
                    &= \theta(n)
    \end{flalign*}

    Implying that we should use rule 2 of the master method. Using rule two,
    we determine the following:

    \begin{flalign*}
        \Call{T}{n} &= \theta(n^{log_b a} \cdot log_2 n) &\\
                    &= \theta(n \cdot log_2 n)
    \end{flalign*}    

\pagebreak
\encircle{4} In a given sequence of n numbers in an array, A[1], A[2], A[3], ..., A[n],
a number may occur multiple times. A number is called the dominating number if it occurs 
strictly more than n/2 times. A sequence may have no dominating number. You may compare
two numbers, A[i] and A[j], \(1 \leq i < j \leq n\), to see if A[i] equals A[j]. However,
we assume this comparison does not tell which one is smaller in case they are not identical.
Please use this kind of comparison among the numbers in array A and counting to determine
if array A has a dominating number. You can expect to know which number is smaller or 
larger if the two numbers in comparison are not both from array A. Your algorithm must use
divide-and-conquer and have \(\mathcal{O}(n \cdot lg n)\) complexity.

\begin{algorithm}[H]
    \caption{
        Algorithm to solve dominating number problem using the divide-and-conquer
        approach.
    }
    \label{alg:algorithm-label}
    \begin{algorithmic}[1]
        \Function{FindDominating}{A, p, r}        
        \If{p equals r}
            \State $ count \gets 1 $
            \State $  $
            \State $ \Return \text{ \textit{count} and \textit{}} $
        \Else
            \State $ threshold \gets \ceil{n/2} $
            \State $ mid \gets \floor{(p + r) / 2} $
            \State $ left \gets \Call{FindDominating}{A, p, mid} $
            \State $ right \gets \Call{FindDominating}{A, mid + 1, r} $
            \\

            \\
            \State $ \Return \text{ } $
        \EndIf

        \EndFunction
    \end{algorithmic}
\end{algorithm}

\pagebreak
\encircle{5} Given a sequence of n real numbers, A[1], A[2], A[3], ..., A[n], we wish
to find a subsequence from A[i] to A[j], \(1 \leq i \leq j \leq n\) such that the sum
of all the numbers in this subsequence is maximized. Please design a \(\mathcal{O}(n \cdot lg n\)
divide-and-conquer algorithm to solve this problem. Note that some numbers may be negative
and the value m may be negative. You need to present good pseudocode.

\begin{algorithm}[H]
    \caption{
        Divide-and-conquer algorithm that locates the subsequence
        of A having the greatest sum.
    }
    \label{alg:algorithm-label}
    \begin{algorithmic}[1]
        \Function{MaxSubsequence}{A, p, r}            
            \If{p equals r}                
                \State $ \Return \text{ p, r and A[p]} $
            \Else
                \State $ mid \gets \floor{(p + r)/2} $
                \State $ left \gets \Call{MaxSubsequence}{A, p, mid} $
                \State $ right \gets \Call{MaxSubsequence}{A, mid + 1, r} $
                \State $ crossing \gets \Call{MaxCrossingSubsequence}{A, p, mid, r} $
                \State $ \Return \text{ \(\max(left, crossing, right)\)} $
            \EndIf            
        \EndFunction
    \end{algorithmic}
\end{algorithm}

\begin{algorithm}[H]
    \caption{
        Function that calculates the maximum subsequence that crosses
        the left and right barrier in the recursive function above
        providing total coverage of the algorithm above.
    }
    \label{alg:algorithm-label}
    \begin{algorithmic}[1]
        \Function{MaxCrossingSubsequence}{A, p, mid, r}
            \State $ i \gets mid $
            \State $ j \gets mid + 1 $
            \State $ max \gets A[i] + A[j] $
            \State $ sequenceSum \gets max $
            \\            
            \For{k from mid - 1 to p}
                \State $ sequenceSum \gets sequenceSum + A[k] $
                \If{sequenceSum > max}
                    \State $ max \gets sequenceSum $
                    \State $ i \gets k $
                \EndIf
            \EndFor
            \\
            \For{k from mid + 2 to r}
                \State $ sequenceSum \gets sequenceSum + A[k] $
                \If{sequenceSum > max}
                    \State $ max \gets sequenceSum $
                    \State $ j \gets k $
                \EndIf
            \EndFor
            \\
            \State $ \Return \text{ i, j and max} $            
        \EndFunction
    \end{algorithmic}
\end{algorithm}



\end{document}